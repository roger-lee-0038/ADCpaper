\section{Introduction}

Battery-powered Internet-of-Things (IoT) devices, e.g., wearables, play an important role in people's daily lives.
In recent years, the adoption of deep learning algorithms on IoT devices are becoming increasingly prevalent to enable edge
intelligence, such as machine vision, speech recognition, and robotics applications. However, content-rich deep models are data 
and energy intensive, limiting their application in energy-constrained, battery-operated IoT devices. There is an urgent need to 
tackle energy challenges to support and sustain the rapid expansion of Artificial Intelligence of Things (AIoTs).

It is a known fact that feature-rich content, such as video and voice, has high data redundancy, including data precision, spatial
and temporal resolution. Exploiting such data redundancy
can potentially reduce computation and energy cost~\cite{lubana_digital_2018, zhao_reinforcement-learning-based_2022}. 
To this end, adaptive computing strategies have recently been proposed to improve system energy efficiency. 
Lubana and Dick presented Digital Foveation, an energy-aware machine vision framework that supports energy-efficient adaptive spatial resolution image processing~\cite{lubana_digital_2018}. 
Zhao {\it et al.} proposed a deep reinforcement learning based spatial and temporal adaptive resolution framework for multi-task video analytics~\cite{zhao_reinforcement-learning-based_2022}. 
LiKamWa {\it et al.} proposed a dynamic frequency method to optimize the camera energy efficiency given sensing quality constraints~\cite{likamwa_energy_2013}. These works demonstrate the potential to effectively minimize the computing cost of edge intelligence. 
Unfortunately, computation is only part of the story. Analog-to-Digital Converters (ADCs), as the the first line of the edge 
analytics pipeline, are equipped with fixed-precision conversion capabilities, introducing serious side effects on system energy 
efficiency.  As demonstrated in previous studies, such fixed-precision ADC design may impose significant energy overhead for
data sensing and communication, which may account for 50\%-75\% of the overall energy consumption of 
image sensors~\cite{choi_energyillumination-adaptive_2015,takayanagi_125-inch_2005,kitamura_33-megapixel_2012}. 
In addition, fixed-precision ADCs further limit the power and energy saving potentials of adaptive-precision computing methods~\cite{leibe_xnor-net_2016,li_ternary_2016,park_energy-efficient_2018}.

% and 34\% of the total machine vision analytics pipeline with backend digital hardware acceleration~\cite{likamwa_redeye_2016}.

This work presents an adaptive-precision ADC architecture design to tackle the energy efficiency challenge of edge intelligence. 
The proposed ADC design utilizes a fine-grain power gating strategy to enable efficient and easy-to-implement run-time 
precision adaptation. The proposed method has been applied to widely used column-parallel ADCs, with two ADC design studies targeting 
data-intensive CMOS image processing. Experimental results demonstrate that the proposed method can reduce ADC energy consumption 
by approximately 50\% with only a few required control circuits. This work makes the following contributions. 

%While more and more deep-learning algorithms have harnessed the redundancy of sensory data to improve the energy efficiency through precision-adaptive computing \cite{leibe_xnor-net_2016}\cite{li_ternary_2016}\cite{park_energy-efficient_2018}, 

%of edge devices, are still generally equipped with fixed precision capabilities. Therefore, opportunities have been offered for adaptive-precision tuning within the ADCs to further lower the energy barrier of edge sensing and computation. 
%and implementation with CIM architecture \cite{chiu_4-kb_2020}\cite{karunaratne_-memory_2020}\cite{jung_crossbar_2022} have been extensively studied, 
%can also be precision-adaptively designed for more intelligent edge computing.
%can also be smartly designed for more intelligent edge computing. 

%As the NN models have been able to process data of varying precision for efficient multi-task analysis [xxxx], opportunities have been offered for us to further improve 
%the systems' energy efficiency by taking algorithm-aware adjustments inside the ADCs. Considering the precision (i.e. the quantization bits) is at the heart of an ADC’s energy constraints, 
%making it dynamically adaptive will be promising.

\begin{enumerate}[\IEEEsetlabelwidth{3)}]
\item 
An adaptive-precision ADC architecture is proposed to enable energy-efficient adaptive data analysis pipeline for edge devices. 
The proposed design is equipped with a fine-grain power gating strategy to enable efficient and easy-to-implement run-time precision adaptation.
Circuit optimization techniques are introduced to eliminate potential metastability caused by power gating. 
Additional control signals are carefully reused and effectively minimized, enabling cost-effective and implementation-friendly power-scaling 
capabilities.  
\item 
The proposed adaptive-precision method has been applied to widely used column-parallel ADCs, targeting data-intensive CMOS image 
processing~\cite{kim_11-bit_2021,nie_single_2020,kumagai_14-inch_2018,park_640_2020}. Overall, we demonstrate with two ADC designs, including 
a column-parallel Single-Slope (SS) ADCs~\cite{snoeij_18v_2005,kleinfelder_10000_2001} and a column-parallel Successive Approximation Register (SAR)/SS ADCs~\cite{kim_area-efficient_2016}. 

%	Both of the two architectures are completely built with not only main functional modules but also peripheral circuits including bandgap circuits, bias circuits and necessary buffers. And the ADCs' energy distribution across different modules is analyzed  in detail. 
	%Two case study ADC designs applied in the CISs are presented with different design specifications.
    %according to which a method combining adaptive precision and fine-grained power gating strategies is proposed
    %for more smart data conversion.   

	\item 
	Experimental results demonstrate that the proposed method can reduce ADC energy consumption by approximately 50\% with only a few required control circuits.

	%Although with different design specifications, the proposed method is successfully applied to the two ADC architectures with the same principles, which shows universally. Besides, the strengths and weaknesses of the two different architectures are specifically discussed.
\end{enumerate} 

The remainder of this paper is organized as follows. 
Sect.~\ref{related} presents related works.
Sect.~\ref{architecture} presents the design of the proposed adaptive-precision ADCs, including ADC architecture overview and adaptive-precision implementation. 
%Sect.~\ref{strategy} describes the implementation of the proposed method for the two ADC designs. 
Sect.~\ref{result} reports the evaluation results and Sect.~\ref{discussion} conducts further discussions. 
Finally, Sect.~\ref{conclusion} concludes this paper.
