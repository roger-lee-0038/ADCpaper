\section{Related Works}\label{related}

Low-precision deep-learning algorithms have been extensively studied\cite{leibe_xnor-net_2016,li_ternary_2016,park_energy-efficient_2018} for better energy efficiency. \cite{leibe_xnor-net_2016} introduced Binary Weight Nets (BWN) and XNOR-Nets to achieve
Deep Neural Networks (DNNs) with binary weights and activations. 
With some accuracy compensation strategies, BWN and XNOR-Nets are able to perform low-precision computations with only 0.8\% and 11\% accuracy loss, respectively.
\cite{li_ternary_2016} adopted weights of three values (i.e., -w, 0, w) to further reduce the accuracy loss of computing. Although this requires an additional bit per weight compared to binary weights, the sparsity of the weights can be exploited to save computation and storage resources.
\cite{park_energy-efficient_2018} presented an outlier-aware accelerator performing dense and low-precision (4-bit) computations on most of the data, while efficiently handling a small number of sparse and high-precision outliers.
Although these works are about downstream low-precision algorithms and relatively seprated from the ADCs, the adaptive-precision design for ADCs needs to take the accuracy requirements of the algorithms into consideration. In this paper, we choose 4-bit conversion for the low-precision mode of the ADCs, with a trade off between the energy-saving capabilities and accuracy loss.

There have also been related works in the design of precision-adaptive ADCs\cite{zhu_06_2013,zhu_6--10-bit_2015,el-halwagy_100-mss5-gss_2018}. \cite{zhu_06_2013} and \cite{zhu_6--10-bit_2015} focused on a single SAR ADC with splittable capacitor array for 8-10-bit or 6-10 bit adaptive-precision.
However, a SAR ADC is not suitable for high-throuput-required image processing applications, where the ADCs are usually in the column-parallel architectures and the Single-Slope(SS)
conversion logic is widely adopted to avoid overhead of area and power consumption.  
The ADC in \cite{el-halwagy_100-mss5-gss_2018} employs a reconfigurable time-to-digital converter (TDC) that can be configured as a double sampling SAR-TDC for the high-resolution modes or as a 4× asynchronous time-interleaved flash-TDC for the high sampling rate modes. The ADC supports continuous sampling rate variations and providing 13–5-bit
adaptive-precision with exponential power scaling. This work showed that the time-domain ADC allows for a wide range of reconfigurability, but the proposed implementation requires rather complex extra control circuits. 
In this paper, considering that the extra control circuits in the column-parallel ADCs must be as simple as possible to avoid area overhead and the energy-efficiency is the primary design constraint, we put aside adjusting the ADCs' sampling rate or supply voltage, and focus on achieving implementation-friendly fine-grained power gating strategies between the ADCs'a full-precision and low-precision modes, while the power scaling potential have also been effectively exploited thanks to the time-domain SS conversion logic.

Other works have dynamically adjusted the the camera’s resolution\cite{lubana_digital_2018} or clock frequency\cite{likamwa_energy_2013} for more efficeient image processing, but the precision of ADCs has not been considered as a variable. Besides, there are efforts trying to relieve the design specifications of ADCs by applying low-precision analog computing firstly close to the sensor \cite{likamwa_redeye_2016,chen_asp_2016,liu_ns-cim_2020}. 
But high-precision ADCs and DNN processors are still there for complex tasks. Therefore, adaptive-precision tuning whithin ADCs remains competitive for efficient hardware reuse..
 
