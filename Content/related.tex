\section{Related Works}\label{related}

Ref.~\cite{leibe_xnor-net_2016} \cite{li_ternary_2016}\cite{park_energy-efficient_2018} have developed low-precision intelligent algorithms for improving the energy effeciency. \cite{leibe_xnor-net_2016} introduced Binary Weight Nets (BWN) and XNOR-Nets to realize
Deep Neural Networks (DNNs) with binary weights and activations. 
By multiplying the outputs with a scale factor to recover the dynamic range, keeping the first and last layers at 32-bit floating point precision, and performing normalization to reduce the dynamic range of the activations,
BWN and XNOR-Nets are able to do low-precision computing with only 0.8\% and 11\% accuracy loss, respectively.\cite{li_ternary_2016} performed DNN computation with weights of three values (i.e., -w, 0, w) for further reducing the accuracy loss. Although this requires an additional bit per weight compared to binary weights, the sparsity of the weights can be exploited by zeros to reduce computation and storage cost.\cite{park_energy-efficient_2018} presented an outlier-aware accelerator performing dense and low-precision (4-bit) computations for a majority of data, while efficiently handling a small number of sparse and high-precision outliers.
Although these works are about  downstream low-precision algorithms and relatively seprated from the ADCs, the design of precision-adaptive solution for ADCs can actually be guided according to the precision requirements of the algorithms. In this paper, considering both of energy-saving capabilities and accuracy loss, we choose 4 bits for low-precision data conversion.

There have also been related works on precision-adaptive ADC design \cite{zhu_06_2013}\cite{zhu_6--10-bit_2015}\cite{el-halwagy_100-mss5-gss_2018}. \cite{zhu_06_2013} and \cite{zhu_6--10-bit_2015} focused on a single SAR ADC with capacitor-seprable digital-to-analog array for 8-10-bit or 6-10 bit adaptive-precision.
However, a SAR ADC is not suitable for image processing applications where high throughput is required thus the ADCs are usually in column-parallel style, and the Single-Slope(SS)
architecture is widely adopted to avoid overhead of area and power.  
The ADC in \cite{zhu_6--10-bit_2015} employs a reconfigurable time-to-digital converter (TDC) that can be configured as a double sampling SAR-TDC for the high-resolution modes or as a 4× asynchronous time-interleaved flash-TDC for the high sampling rate modes. The ADC supports continuous sampling rate variations from 100 MS/s to 5 GS/s and providing 13–5-bit
adaptive-precision with exponential power scaling. This work showed that the time-domain nature of the ADC allows for efficient hardware reuse and a wide range of reconfigurability, though the ADC is also not for image processing applications. 
In addition, relatively complex extra control circuits are required in these works to continuously adjust the sampling-rate and precision of the ADCs. However, in the column-parallel ADCs, the extra control circuits have to be as simple as possible for the consistency across columns and area saving. It is why adaptive-sampling-rate and continuously-adjusted-precision have not been considered in this paper. Instead, we focus on exploiting the power scaling potential from the time-domain SS conversion logic between a full-precision mode and a low-precision mode, with implementation-friendly and fine-grained power gating strategies.

Other works have dynamically adjusted the the camera’s resolution\cite{lubana_digital_2018} or clock frequency\cite{likamwa_energy_2013}for more efficeient image processing, but the precision of ADCs has not been considered as a variable. Besides, there are efforts trying to relieve the design specifications of ADCs by applying low-precision analog computing firstly close to the sensor \cite{likamwa_redeye_2016}\cite{chen_asp_2016}\cite{liu_ns-cim_2020}. 
But high-precision ADCs and DNN processors are still there for complex tasks. Therefore, taking adaptive-precision adjustments inside ADCs remains competitive.
 
